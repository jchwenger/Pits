\documentclass[9pt, twocolumn]{memoir}
\usepackage{lmodern}
\usepackage{amssymb,amsmath}
\usepackage{ifxetex,ifluatex}
\usepackage{fixltx2e} % provides \textsubscript
\ifnum 0\ifxetex 1\fi\ifluatex 1\fi=0 % if pdftex
  \usepackage[T1]{fontenc} 
  \usepackage[utf8]{inputenc}
\else % if luatex or xelatex
  \ifxetex
    \usepackage{mathspec}
  \else
    \usepackage{fontspec}
  \fi
  \defaultfontfeatures{Ligatures=TeX,Scale=MatchLowercase}
\fi
% use upquote if available, for straight quotes in verbatim environments
\IfFileExists{upquote.sty}{\usepackage{upquote}}{}
% use microtype if available
\IfFileExists{microtype.sty}{%
\usepackage{microtype}
\UseMicrotypeSet[protrusion]{basicmath} % disable protrusion for tt fonts
}{}
\usepackage{hyperref}
\hypersetup{colorlinks,urlcolor=blue}
\renewcommand{\abstractname}{\vspace{-\baselineskip}}
\hypersetup{unicode=true,
            pdfborder={0 0 0},
            breaklinks=true}
\urlstyle{same}  % don't use monospace font for urls
\IfFileExists{parskip.sty}{%
\usepackage{parskip}
}{% else
\setlength{\parindent}{0pt}
\setlength{\parskip}{6pt plus 2pt minus 1pt}
}
\setlength{\emergencystretch}{3em}  % prevent overfull lines
\providecommand{\tightlist}{%
  \setlength{\itemsep}{0pt}\setlength{\parskip}{0pt}}
\setcounter{secnumdepth}{0}
\setlength\parindent{0pt}
% Redefines (sub)paragraphs to behave more like sections
\ifx\paragraph\undefined\else
\let\oldparagraph\paragraph
\renewcommand{\paragraph}[1]{\oldparagraph{#1}\mbox{}}
\fi
\ifx\subparagraph\undefined\else
\let\oldsubparagraph\subparagraph
\renewcommand{\subparagraph}[1]{\oldsubparagraph{#1}\mbox{}}
\fi

\title{{\Huge Recursus}  \\ {\large \href{http://recursus.co/}{www.recursus.co}} \\ {\large Technical Specifications}}
\date{{\normalsize 2018}}
\author{Jérémie C. Wenger}

\begin{document}
\maketitle
\thispagestyle{empty}

For the MA/MFA End of Year Show
`\href{http://echosystems.xyz/}{Echosytems}', prints were ordered at
\href{https://www.digitalarte.co.uk/}{digitalarte} in Greenwich with the
following dimensions: 
\\

WordSquares:

\begin{itemize}
\tightlist
\item
  21 x 21 cm
\end{itemize}

Subwords:

\begin{itemize}
\tightlist
\item
  short: 29.7 x 8.91 cm
\item
  long: 44.55 x 8.91 cm (`microspectrophotometrical(ly)')
\end{itemize}

AIT:

\begin{itemize}
\tightlist
\item
  `I am the most in': 59.4 x 16.5 cm
\item
  `I am the new ones': 42 x 75.6 cm
\item
  `Insane not enough to expect me': 59.4 x 16.5 cm
\item
  `So weak resist resist': 29.7 x 29.7
\item
  `The self-torture is the strange thing': 59.4 x 33
\end{itemize}


The Paper used: `Hahnemühle Photo Rag 308gsm'


Total cost (with 10\% student discount): £198.72

~

\fancybreak{§}

~

The wire frame \& other tools were bought at \href{http://www.flints.co.uk/content/}{Flints}:

\begin{itemize}
\tightlist
\item
  100 m Lifting-3mm 6x19 Galv Wire Rope
\item
  100 x 3-4 mm Wire Rope Grips Din741
\item
  7mm Wera Kraftform 395 Nutspinner 
\end{itemize}

Total cost: £60.42

\vfill\null
\newpage

The possibilities for exhibiting these texts are quite numerous, and almost any of the parameters at hand can be modified to suit the space and occasion. Sizes here are only an indication, and were chosen mainly for readability purposes. One could imagine smaller prints, for a very intimate setting, and somewhat larger ones as well, especially for the prose texts, although the `monumental', e.g. one square occupying an entire wall, is to be avoided. 

\

The setting for the Goldsmiths show consisted in steel ropes, 14 in total, hung from a pipe at a height of roughly 3.5 m, and stabilised by a steel beam at the bottom. Squares were simply taped at the back, although this method could certainly be improved (possibly with eyelets and rivets in the corners, or with horizontal ropes and grippers of some kind. The intention was to make visible the tension between, on the one hand, the lightness and fragility of the paper, and the rigid (if still somewhat flexible) metallic frame, a loose image for the computational constraints used in the literary process. 

\

Other layouts could certainly be envisaged, such as: metallic ropes thinner than the 3mm ones used (although nylon wire should be avoided); each piece hung individually from the ceiling in the centre of the space, investing an entire room and allowing people to walk between them; tables or other horizontal surfaces on which the prints would be laid, creating a context in which the pieces would be looked at as manuscripts rather than art works.

\

Two spots were hung on either side, higher than the whole installation, with the same warm light bulbs. The whole room was otherwise unlit and did not contain other sound-emitting works, creating an atmosphere conducive to attention and reading. Just as for other parameters, the flexibility of the piece, the purpose of which is the texts more than the setting, the lighting could be changed to something more direct and bright (as found in many art galleries).

\end{document}
